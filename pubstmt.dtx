% \iffalse meta-comment
% 
% pubstmt.ins
% 
% Copyright (C) 2017 by Thai Son Hoang
% <T dot S dot Hoang at ecs dot soton dot ac dot uk>
% --------------------------------------------------------------------
% 
% This file may be distributed and/or modified under the
% conditions of the LaTeX Project Public License, either version 1.3c
% of this license or (at your option) any later version.
% The latest version of this license is in:
% 
%      http://www.latex-project.org/lppl.txt
% 
% and version 1.3c or later is part of all distributions of LaTeX 
% version 2008/05/04 or later.
% 
% This work has the LPPL maintenance status "author-maintained".
% 
% The Current Maintainer of this work is T.S. Hoang.
%
% This work consists of the files pubstmt.dtx, pubstmt.ins,
% the derived file pubstmt.sty, the generated documentation
% pubstmt.pdf, and some sample documents.
% 
% \fi
% 
% \iffalse
%<pubstmt>\NeedsTeXFormat{LaTeX2e}\relax
%<pubstmt>\ProvidesPackage{pubstmt}
%<pubstmt>    [2017/11/17 v1.0 Package for typesetting information linking to publisher's website] 
% 
%<*driver> 
\documentclass[a4paper]{ltxdoc}
\usepackage{pubstmt}
\EnableCrossrefs
% ^^A\CodelineIndex
\PageIndex
\RecordChanges

\begin{document}
\DocInput{pubstmt.dtx}
\end{document}
%</driver>
% \fi
% 
% \CheckSum{56}
% 
% \CharacterTable
% {Upper-case    \A\B\C\D\E\F\G\H\I\J\K\L\M\N\O\P\Q\R\S\T\U\V\W\X\Y\Z
% Lower-case    \a\b\c\d\e\f\g\h\i\j\k\l\m\n\o\p\q\r\s\t\u\v\w\x\y\z
% Digits        \0\1\2\3\4\5\6\7\8\9
% Exclamation   \!     Double quote  \"     Hash (number) \#
% Dollar        \$     Percent       \%     Ampersand     \&
% Acute accent  \'     Left paren    \(     Right paren   \)
% Asterisk      \*     Plus          \+     Comma         \,
% Minus         \-     Point         \.     Solidus       \/
% Colon         \:     Semicolon     \;     Less than     \<
% Equals        \=     Greater than  \>     Question mark \?
% Commercial at \@     Left bracket  \[     Backslash     \\
% Right bracket \]     Circumflex    \^     Underscore    \_
% Grave accent  \`     Left brace    \{     Vertical bar  \|
% Right brace   \}     Tilde         \~}
% 
% 
% \changes{v1.0}{2017/11/17}{Initial version}
% 
% \GetFileInfo{pubstmt.sty}
% 
% \DoNotIndex{\\}
% \DoNotIndex{\DeclareOption}
% \DoNotIndex{\ProcessOptions}
% \DoNotIndex{\RequirePackage}
% \DoNotIndex{\arabic}
% \DoNotIndex{\begin}
% \DoNotIndex{\csname,\csuse}
% \DoNotIndex{\def,\do,\dolistloop}
% \DoNotIndex{\end,\endcsname,\expandafter}
% \DoNotIndex{\hline}
% \DoNotIndex{\ifstrequal,\iftoggle,\item}
% \DoNotIndex{\label,\labelformat,\listadd}
% \DoNotIndex{\medskip}
% \DoNotIndex{\newcommand,\newcounter,\newenvironment,\newtoggle,\nomenclature}
% \DoNotIndex{\quad}
% \DoNotIndex{\renewcommand,\renewenvironment,\ref,\refstepcounter}
% \DoNotIndex{\setcounter,\small}
% \DoNotIndex{\textsf,\textwidth,\togglefalse,\toggletrue}
% \DoNotIndex{\value}
% \DoNotIndex{\xspace,\vspace,\hspace}
% \DoNotIndex{\copyright,\url}
%
% \title{The \textsf{pubstmt} package\thanks{This document
% corresponds to \textsf{pubstmt}~\fileversion, dated~\filedate.}}
% \author{Thai Son Hoang \\ ECS, University of Southampton \\ \texttt{<T dot S dot Hoang at ecs dot
% soton dot ac dot uk>}}
% \date{November 16, 2017}
% 
% \maketitle
% 
% ^^A %%%%% Abstract %%%%%
% \begin{abstract}
%   This package provides utilities for typesetting information
%   linking to publisher website.  It was developed at the University
%   of Southampton.
% \end{abstract}
% 
% ^^A %%%%% Table of contents %%%%%
% \tableofcontents
% 
% ^^A %%%%% Introduction %%%%%
% \section{Introduction}
% 
% This package was developed in order to ease the typesetting
% information linking to publisher website (the publisher's statement).
% 
% ^^A %%%%% Usage %%%%%%
% \section{Usage}
% 
% Just like any other package, you need to request this package with a
% |\usepackage| command in the preamble.
%
% So in the simpler case, one just types
% 
% \indent |\usepackage{pubstmt}|
%
% \noindent to load the package.  The information related to the
% publication, e.g., publisher's DOI number, its book title and
% publisher's name can be set afterwards.
% \begin{verbatim}
% \pubstmtSetDOI{10.1007/978-3-319-68499-4_5}
% \pubstmtSetBookTitle{CONF2017: Proceedings of some conference}
% \pubstmtSetPublisher{Springer}
% \end{verbatim}
% The publisher statement can be typeset on any page using |\thispagestyle{pubstmt}|.
% \StopEventually{
% \PrintChanges
% \PrintIndex
% }
%   
% ^^A %%%%% Implementation %%%%%
% \section{Implementation}
%
% \subsection{Package Loading}
% \label{sec:package-loading}
% The implementation is quite straightforward.  Our implementation is
% based on the |fancyhdr| package.  We also require |url| package for
% typesetting the URLs.
% 
% \iffalse ^^A BEGIN Produce comments only in the resulting style file
%<pubstmt>
%<pubstmt>%%%%% BEGIN Package loading %%%%%
% \fi ^^A END Produce comments only in the resulting style file
%
%    \begin{macrocode}
\RequirePackage{fancyhdr}
\RequirePackage{url}
\RequirePackage{hyperref}
%    \end{macrocode}
%
% \iffalse ^^A BEGIN Produce comments only in the resulting style file
%<pubstmt>%%%%% END Package loading %%%%%
%<pubstmt>
% \fi ^^A END Produce comments only in the resulting style file
%
% \subsection{Internal Helper Macros}
% \label{sec:intern-help-macr}
% We define with some internal helper macros that can be (re-)set by
% the users or package options.
% \iffalse ^^A BEGIN Produce comments only in the resulting style file
%<pubstmt>
%<pubstmt>%%%%% BEGIN Internal Helper Macros %%%%%
%<pubstmt>% ========================
% \fi ^^A END Produce comments only in the resulting style file
%
% \begin{macro}{\pubstmt@doi}
%   \changes{v1.0}{2017/11/16}{Initial version}
%   We define a macro for the publisher's DOI number corresponding to
%   the publication.  The users are expected to set this DOI number
%   using the command |\pubstmtSetDOI| (defined later).
% \iffalse ^^A BEGIN Produce comments only in the resulting style file
%<pubstmt>% Publisher's DOI number
% \fi ^^A END Produce comments only in the resulting style file
%    \begin{macrocode}
\newcommand{\pubstmt@doi}{DOI number}

%    \end{macrocode}
% \end{macro}
%
% \begin{macro}{\pubstmtdoi}
%   \changes{v1.0}{2017/11/17}{Initial version}
%   For convenient, we make a public command to allow the users to
%   directly use the publisher's DOI number, for example, when
%   defining his own publisher statement using |pubstmtSetStatement|
%   (defined later).
% \iffalse ^^A BEGIN Produce comments only in the resulting style file
%<pubstmt>% (public) Publisher's DOI number
% \fi ^^A END Produce comments only in the resulting style file
%    \begin{macrocode}
\newcommand{\pubstmtdoi}{\pubstmt@doi}

%    \end{macrocode}
% \end{macro}
%
% \begin{macro}{\pubstmtSetDOI}
%   \changes{v1.0}{2017/11/16}{Initial version}
%   We define the command to (re-)set the DOI number as follows.
%    \begin{macrocode}
%<pubstmt>% Command to (re-)set the publication's DOI number.
%<pubstmt>%
%<pubstmt>% Arguments:
%<pubstmt>% 1. The DOI number
%<pubstmt>%
%<pubstmt>% Usage:
%<pubstmt>% - \pubstmtSetDOI{10.1007/978-3-319-68499-4_5} will set the
%<pubstmt>% publication DOI number accordingly.
\newcommand{\pubstmtSetDOI}[1]{%
  \renewcommand{\pubstmt@doi}{#1}%
}%

%    \end{macrocode}
% \end{macro}
%
% \begin{macro}{\pubstmt@booktitle}
%   \changes{v1.0}{2017/11/16}{Initial version}
%   Similarly we define a macro for the book title of the publication.
%   The users are expected to set this book title using the command
%   |\pubstmtSetBookTitle| (defined later).
% \iffalse ^^A BEGIN Produce comments only in the resulting style file
%<pubstmt>% Publication's book title
% \fi ^^A END Produce comments only in the resulting style file
%    \begin{macrocode}
\newcommand{\pubstmt@booktitle}{Book title}

%    \end{macrocode}
% \end{macro}
%
% \begin{macro}{\pubstmtbooktitle}
%   \changes{v1.0}{2017/11/17}{Initial version}
%   For convenient, we make a public command to allow the users to
%   directly use the publication's book title, for example, when
%   defining his own publisher statement using |pubstmtSetStatement|
%   (defined later).
% \iffalse ^^A BEGIN Produce comments only in the resulting style file
%<pubstmt>% (public) Publication's book title
% \fi ^^A END Produce comments only in the resulting style file
%    \begin{macrocode}
\newcommand{\pubstmtbooktitle}{\pubstmt@booktitle}

%    \end{macrocode}
% \end{macro}
%
% \begin{macro}{\pubstmtSetBookTitle}
%   \changes{v1.0}{2017/11/16}{Initial version}
%   The macro for setting the book title is defined as follows.
%    \begin{macrocode}
%<pubstmt>% Command to (re-)set the publication's book title.
%<pubstmt>%
%<pubstmt>% Arguments:
%<pubstmt>% 1. The book title
%<pubstmt>%
%<pubstmt>% Usage:
%<pubstmt>% - \pubstmtSetBookTitle{CONF2017: Proceedings of
%<pubstmt>% some conference} will set the publication book title to
%<pubstmt>% "CONF2017: Proceedings of some conference". 
\newcommand{\pubstmtSetBookTitle}[1]{%
  \renewcommand{\pubstmt@booktitle}{#1}
}%

%    \end{macrocode}
% \end{macro}
%
% \begin{macro}{\pubstmt@publisher}
%   \changes{v1.0}{2017/11/16}{Initial version}
%   Similarly we define a macro for publisher of the publication.
%   The users are expected to set this publisher using the command
%   |\pubstmtSetPublisher| (defined later).
% \iffalse ^^A BEGIN Produce comments only in the resulting style file
%<pubstmt>% Publication's publisher
% \fi ^^A END Produce comments only in the resulting style file
%    \begin{macrocode}
\newcommand{\pubstmt@publisher}{Publisher}

%    \end{macrocode}
% \end{macro}
%
% \begin{macro}{\pubstmtpublisher}
%   \changes{v1.0}{2017/11/17}{Initial version}
%   For convenient, we make a public command to allow the users to
%   directly use the publication's publisher, for example, when
%   defining his own publisher statement using |pubstmtSetStatement|
%   (defined later).
% \iffalse ^^A BEGIN Produce comments only in the resulting style file
%<pubstmt>% (public) Publication's publisher
% \fi ^^A END Produce comments only in the resulting style file
%    \begin{macrocode}
\newcommand{\pubstmtpublisher}{\pubstmt@publisher}

%    \end{macrocode}
% \end{macro}
%
% \begin{macro}{\pubstmtSetPublisher}
%   \changes{v1.0}{2017/11/16}{Initial version}
%   The macro for setting the publisher is defined as follows.
%    \begin{macrocode}
%<pubstmt>% Command to (re-)set the publication's publisher.
%<pubstmt>%
%<pubstmt>% Arguments:
%<pubstmt>% 1. The publisher
%<pubstmt>%
%<pubstmt>% Usage:
%<pubstmt>% - \pubstmtSetPublisher{Springer} will set the publication's
%<pubstmt>% publisher to "Springer"
\newcommand{\pubstmtSetPublisher}[1]{%
  \renewcommand{\pubstmt@publisher}{#1}
}%

%    \end{macrocode}
% \end{macro}
%
% \begin{macro}{\pubstmt@statement}
%   \changes{v1.0}{2017/11/16}{Initial version}
%   Finally, we define the command for publisher's statement. By
%   default this is the Springer's style statement. The users can
%   (re-)set the statement using |\pubstmtSetStatement| (defined
%   later).
% \iffalse ^^A BEGIN Produce comments only in the resulting style file
%<pubstmt>% Publication's publisher
% \fi ^^A END Produce comments only in the resulting style file
%    \begin{macrocode}
\newcommand{\pubstmt@statement}{%
  \vspace{-10ex}
  The original publication is available at \url{http://doi.org/\pubstmt@doi}\\%
  In \pubstmt@booktitle{} $\copyright$ \pubstmt@publisher%
}%

%    \end{macrocode}
% \end{macro}
%
% \begin{macro}{\pubstmtSetStatement}
%   \changes{v1.0}{2017/11/16}{Initial version}
%   The macro for setting the publisher statement is defined as follows.
%    \begin{macrocode}
%<pubstmt>% Command to (re-)set the publisher statement.
%<pubstmt>%
%<pubstmt>% Arguments:
%<pubstmt>% 1. The statement
%<pubstmt>%
%<pubstmt>% Usage:
%<pubstmt>% - \pubstmtSetStatement{Some statement} will set the 
%<pubstmt>% publisher statement accordingly.
\newcommand{\pubstmtSetStatement}[1]{%
  \renewcommand{\pubstmt@statement}{#1}
}%

%    \end{macrocode}
% \end{macro}
% \iffalse ^^A BEGIN Produce comments only in the resulting style file
%<pubstmt>
%<pubstmt>%%%%% END Internal Helper Macros %%%%%
% \fi ^^A END Produce comments only in the resulting style file
%
%
% \subsection{The \textrm{pubstmt} page style}
% \label{sec:pubstmt-page-style}
% We define a new fancy page style called |pubstmt| as follows.
% 
% \iffalse^^A BEGIN Produce comments only in the resulting style file
%<*pubstmt>

% *pubstmt* fancy page style.  With this page style, all header fields
% is clear and the information statement about the publisher website
% is inserted as the centre field of the header.
%
% Usage:
% - Typically for the first page of the article, \thispagestyle{pubstmt}
%</pubstmt>
% \fi^^A END Produce comments only in the resulting style file
%    \begin{macrocode}
\fancypagestyle{pubstmt}
{
  \fancyhead{} % Clear all header fields
  \fancyhead[C]{%
    \pubstmt@statement
   }%
  \renewcommand{\headrulewidth}{0 pt} % No header rule
}

%%%%% BEGIN Declaration of options %%%%%
% ========================
\DeclareOption{llncs}{%
}%

%%%%% END Declaration of options %%%%%
% ========================

%%%%% BEGIN Execution of options %%%%%
% ========================

\ProcessOptions
%%%%% END Execution of options %%%%%

%    \end{macrocode}
%
% \Finale
\endinput